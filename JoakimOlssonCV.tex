\documentclass{resume}

% bibliography with multiple entries
\usepackage{multibib}
\newcites{papers}{Selected Publications}
\newcites{talks}{Selected Talks}

\usepackage{xpatch}
\makeatletter
\newcommand\removebibheader
  {\xpatchcmd\std@thebibliography
    {\section*{\refname}%
     \@mkboth{\MakeUppercase\refname}{\MakeUppercase\refname}%
    }{}{}{}%
  }
\makeatother

\usepackage{etoolbox}
\BeforeBeginEnvironment{thebibliography}{%
  \let\origsection\section% save original definition of \section
  \let\section\subsection%  make \section behave like \subsection
}
\AfterEndEnvironment{thebibliography}{%
  \let\section\origsection% restore original definition of \section
}

\usepackage[left=0.75in,top=0.6in,right=0.75in,bottom=0.6in]{geometry}

\newcommand{\CC}{C\nolinebreak\hspace{-.05em}\raisebox{.4ex}{\tiny\bf +}\nolinebreak\hspace{-.10em}\raisebox{.4ex}{\tiny\bf +}}
\def\CC{{C\nolinebreak[4]\hspace{-.05em}\raisebox{.4ex}{\tiny\bf ++}}}

\usepackage{fancyhdr}
\pagestyle{fancy}
\renewcommand{\headrulewidth}{0pt}
\newcommand{\COMMITHASH}{TRAVISCOMMITHASH}
\newcommand{\BUILDNUMBER}{TRAVISBUILDNUMBER}

\fancyhead{}
\fancyfoot{}
\rfoot{\footnotesize \href{{https://jmrolsson.github.io/cv/JoakimOlssonCV.pdf}}{\faLink~Joakim Olsson's CV} (version: \href{https://github.com/jmrolsson/cv/tree/\COMMITHASH}{\COMMITHASH})}

\name{\href{http://jmrolsson.com}{Joakim Olsson}}
\address{University of Chicago, PRC~161 \hfill CERN CH - B\^atiment 1-R-013}
\address{5640 S Ellis Ave, Chicago IL 60637 \hfill Gen\`eve, Switzerland}
\address{\href{tel:+13127809260}{\faicon{phone}~+1 (312) 780-9260} \hfill \href{https://orcid.org/0000-0003-4154-8139}{\faicon{key}~ORCID: 0000-0003-4154-8139}}
\address{\href{mailto:olsson@uchicago.edu}{\faicon{envelope}~olsson@uchicago.edu} \hfill \href{https://www.twitter.com/jmrolsson}{\faicon{twitter}} \href{https://www.linkedin.com/in/jmrolsson}{\faicon{linkedin}} \href{https://www.github.com/jmrolsson}{\faicon{github}~@jmrolsson} \href{http://jmrolsson.com}{\faicon{link}~jmrolsson.com}\vspace{1.0em}}

\linespread{1.0}

\begin{document}

%----------------------------------------------------------------------------------------
% RESEARCH INTERESTS	
%----------------------------------------------------------------------------------------

\begin{rsection}{Research Interests}
  Experimental high energy physics and cosmology, jet substructure and hadronic final states, particle detector hardware development and testing, software development, algorithms, machine learning applications in high energy physics
\end{rsection}

%----------------------------------------------------------------------------------------
%	EDUCATION
%----------------------------------------------------------------------------------------

\begin{rsection}{Education}

  \rsubsectiontitle{University of Chicago}{Chicago, IL}{Ph.D. Physics, Advisor: David W. Miller}{September 2012 -- July 2018 (expected)}\\
  \rsubsubsectiontitle{M.Sc. Physics}{2014}
  \vspace{-0.2em}
  \item Dissertation: \\{\em Searching for Supersymmetry in Fully-Hadronic Final States with the ATLAS Experiment}

%------------------------------------------------

  \rsubsectiontitle{Chalmers University of Technology}{Gothenburg, Sweden}{M.Sc. Fundamental Physics}{2010 -- 2012}
  \vspace{-0.2em}
  \item Thesis: {\em Simulation of physics beyond the Standard Model with an extra $U(1)'$ gauge boson}

%------------------------------------------------

  \rsubsectiontitle{Tohoku University}{Sendai, Japan}{Exchange student: full time research in lab, courses in physics and Japanese}{2009 -- 2010}

%------------------------------------------------

  \rsubsectiontitle{Chalmers University of Technology}{Gothenburg, Sweden}{B.Sc. Engineering Physics}{2006 -- 2009}
  \vspace{-0.2em}
  \item Thesis: {\em Autonomous guidance system for a model airplane}

\end{rsection}

%----------------------------------------------------------------------------------------
%	RESEARCH
%----------------------------------------------------------------------------------------

\begin{rsection}{Relevant Experience}

  \begin{rsubsection}{CERN (European Organization for Nuclear Research)}{Geneva, Switzerland}{Ph.D. Student, ATLAS Experiment}{September 2012 -- Present}
  \item Lead analyser in two searches for supersymmetry with the ATLAS detector, an 8 TeV search for R-parity-violating stops \cite{RPVstop2016}, and a 13 TeV search for direct production of a chargino and a neutralino decaying via Wh to fully-hadronic final states (paper in-progress). I was responsible for all aspects of the analysis, including private sample production for sensitivity studies, developing the strategy to suppress background, data/MC comparisons, systematics, HistFitter limit setting, helping to maintaining the analysis framework, editing supporting note, etc.
  \item Analyzed data and helped to prepare material for an ATLAS Run I search for R-parity-violating supersymmetric gluino pair-production with signatures based on high jet multiplicities \cite{Multijet2015}. 
  \item Measured the ATLAS calorimeter response to single isolated charged hadrons (E/p) during LHC Run II data taking, focusing primarily on the ATLAS Tile Calorimeter (TileCal).
  \item Developed software, analyzed data, supervised students, and took on a leading role during four test beam runs at CERN for the ATLAS TileCal High Luminosity LHC upgrade.
  \item Oversaw the trigger and beam line elements during TileCal test beam activities at CERN. 
  \item Led a small team responsible for data quality of TileCal (Tile DQ Team Leader).
  \item Supervised students working in the University of Chicago ATLAS group at CERN.
  \item Initiated a project using Deep Neural Networks for ATLAS calorimeter topo-cluster classification.
  \end{rsubsection}

%------------------------------------------------

  \begin{rsubsection}{University of Chicago, Kavli Institute for Cosmological Physics}{Chicago, IL}{Graduate Researcher for the South Pole Telescope (SPT-3g)}{September 2012 -- May 2013}
  \item Measured loss properties in Nb superconducting microstrip transmission lines for coupling the receiving antenna to the transition-edge sensor (TES) in each pixel of the SPT-3g focal plane.
  \item Designed a special ''cold stage'' in Solid Works to minimize heat transfers and reduce noise. 
  \item Worked for Professor \href{https://kicp.uchicago.edu/people/profile/john_carlstrom.html}{John Carlstrom} as part of a course in Advanced Experimental Physics.
  \end{rsubsection}

%------------------------------------------------

  \begin{rsubsection}{CERN (European Organization for Nuclear Research)}{Geneva, Switzerland}{Summer Student in the Caltech CERN CMS Group}{June -- September, 2011}
  \item Studied the impact of spurious signals (''spikes'') in the avalanche photo diodes in the CMS Electromagnetic Calorimeter (ECAL). Authored an CMS Draft Analysis Note: CMS AN-11-481.
  \item Presented my work \href{http://indico.cern.ch/conferenceDisplay.py?confId=135576}{(indico link)} for a large audience at end of the program.
  \item Worked under the supervision of Professor \href{http://www.hep.caltech.edu/~smaria/}{Maria Spiropulu}.
  \end{rsubsection}

%------------------------------------------------

  \begin{rsubsection}{Tohoku University}{Sendai, Japan}{Graduate Researcher in the Solid-State Quantum Transport Group}{October 2009 -- August 2010}
  \item Investigated fundamental characteristics of quantum point contacts; measuring quantized conductance, quantum Hall effect, and resistively-detected Nuclear Magnetic Resonance.
  \item Verified and improved upon research published by the group (PRL 100, 186801), where \textit{e-e} interactions was suggested as an influence for deviations from idealized conductance values.
  \item Improved the accuracy in a transport measurement system by significant noise reduction.
  \item Worked under the supervision of Professor \href{http://quant-trans.org/lab/profile-e.html}{Yoshiro Hirayama}.
  \end{rsubsection}

%------------------------------------------------

  \begin{rsubsection}{University of Florida}{Gainesville, FL}{REU at the Institute for High Energy Physics and Astrophysics}{June -- August, 2009}
  \item Implemented a track reconstruction algorithm for the CMS muon system (using $\CC$ and ROOT). 
  \item Developed trip data analysis software for the CMS Muon Endcap High-Voltage System.
  \item Worked under the supervision of Professor \href{http://www.phys.ufl.edu/~mitselmakher/}{Guenakh Mitselmakher} and Professor \href{http://www.phys.ufl.edu/faculty/furic.shtml}{Ivan Furic}.
  \end{rsubsection}

\end{rsection}

%----------------------------------------------------------------------------------------
%	TEACHING
%----------------------------------------------------------------------------------------

\begin{rsection}{Teaching}

  \rsubsectiontitle{University of Chicago}{Chicago, IL}{}{}\\[0.5em]
  \begin{rsubsubsection}{Graduate Student Teaching Assistant}{2012 -- 2013, Winter and Spring 2015, Winter 2017}
  \item Demonstrated physics problem solving in discussion sessions of 15-80 students, graded homework, and prepared solutions for courses in classical physics, modern physics, and particle physics.
  \item Supervised experiments and graded technical reports in senior undergraduate physics labs.
  \item Led hands-on electronics laboratory classes for about $20$ students.
  \item Courses: \vspace{-0.1cm}
    \begin{itemize}
      \item PHYS154 -- Modern Physics \hfill Fall 2012 \vspace{-0.2cm}
      \item PHYS185 -- Classical Physics \hfill Winter 2013 \vspace{-0.2cm}
      \item PHYS237 -- Particle Physics \hfill Spring 2013 \vspace{-0.2cm}
      \item PHYS211 -- Advanced Undergraduate Labs \hfill Winter 2015 \vspace{-0.2cm}
      \item PHYS226 -- Electronics \hfill Spring 2015 \vspace{-0.2cm}
      \item PHYS211 -- Advanced Undergraduate Labs \hfill Winter 2017 \vspace{-0.2cm}
    \end{itemize}
  \end{rsubsubsection}

\end{rsection}

\newpage

%----------------------------------------------------------------------------------------
%	SELECTED PUBLICATIONS
%----------------------------------------------------------------------------------------

\begin{rsection}{Selected Publications}
\bibliographystylepapers{atlasBibStyleWithTitle}
{\removebibheader
\bibliographypapers{papers}
\nocitepapers{*}
}
\end{rsection}

%----------------------------------------------------------------------------------------
%	SELECTED TALKS
%----------------------------------------------------------------------------------------

\begin{rsection}{Selected Talks}
\bibliographystyletalks{atlasBibStyleWithTitle}
{\removebibheader
\bibliographytalks{talks}
\nocitetalks{*}
}
\end{rsection}

%----------------------------------------------------------------------------------------
%	HONORS AND AWARDS
%----------------------------------------------------------------------------------------

\begin{rsection}{Honors \& Awards}
  Scholarship for Excellent Academic Achievements, Hvitfeldtska, Gothenburg, Sweden, 2011\\%
  Scholarship for Studies in Japan, Adlerbertska Foundation, Gothenburg, Sweden, 2009\\%
  Scholarship for Studies in Japan, Sweden-Japan Foundation, Stockholm, Sweden, 2009\\%
  Scholarship for Studies in Japan, Japan Students Services Organization, Sendai, Japan, 2009%
\end{rsection}

%----------------------------------------------------------------------------------------
%	LEADERSHIP 
%----------------------------------------------------------------------------------------

\begin{rsection}{Leadership}

  \begin{rsubsection}{Intize.org}{Gothenburg, Sweden}{Math tutor for high-school students}{January 2011 -- June 2012}
  \item Mentored a group of Swedish high school students for a few hours every week.  
  \end{rsubsection}

%------------------------------------------------

  \begin{rsubsection}{CETAC (Chalmers Engineering Trainee Appointment Committee)}{Gothenburg, Sweden}{Appointment Manager and Member of the Board}{2008 -- 2009}
	\item Coordinated trainee appointments for 13 Swedish engineering students at companies in the US and Canada. Despite the 2009 financial crisis, we managed to find paid positions for all members. 
  \end{rsubsection}

%------------------------------------------------

  \begin{rsubsection}{Employment Market Group at the Physics Student Union}{Gothenburg, Sweden}{Chairman}{2007 -- 2008}
		\item Organized a field-trip to CERN for 35 undergraduate students in Engineering Physics at Chalmers.
  \end{rsubsection}

\end{rsection}

%----------------------------------------------------------------------------------------
% OUTREACH	
%----------------------------------------------------------------------------------------

\begin{rsection}{Outreach}
  \textbf{The Museum of Science and Industry, Chicago (March 2013)}\\
  Gave a popular science talk for the general public: The Higgs Boson -- What's the big deal? A HUGE discovery of a TINY particle!\\[0.1cm]
  \textbf{Public Event: Screening of Particle Fever and Discussion Panel (August 2015)}\\
  Chaired a discussion panel of physicists after a movie screening for the general public.
\end{rsection}

\newpage

%----------------------------------------------------------------------------------------
%	SKILLS/INFO
%----------------------------------------------------------------------------------------

\begin{rsection}{Skills/Info}

  \begin{tabular}{ @{} >{\bfseries}l @{\hspace{6ex}} l }
    Programming & $C$, $\CC$, Python, Bash, Java, JavaScript, Ruby \\
    Markup &  JSON, YAML, XML, HTML, CSS, \LaTeXe  \\
    Software/Tools & Unix/Linux, Version Control, \href{https://github.com/jmrolsson}{GitHub}/\href{https://gitlab.cern.ch/jolsson}{GitLab}, Vim (awesome editor), Emacs \\
    & ROOT, Keras, TensorFlow, scikit-learn, NumPy, SciPy, Matplotlib, pandas \\ 
    & root\_numpy, rootpy, PyROOT, Docker, AutoCAD, SolidWorks, LabVIEW \\
    & MATLAB, Mathematica, Adobe Illustrator/Photoshop/InDesign \\ 
    Hardware & Calorimeters, Arduino controllers, NIM crates, FPGA basics, soldering\\ 
    Interpersonal & Project management, leadership, mentorship, public speaking, science outreach\\ 
    Languages & English (bilingual), Swedish (native), Japanese (elementary), French (elementary) \\
    Other & Swedish citizen, Swedish and US driver's license \\
    Hobbies & Skiing, long distance running, powerlifting, hiking, wave surfing \\
    & traveling, reading, coding
  \end{tabular}

\end{rsection}
\end{document}
